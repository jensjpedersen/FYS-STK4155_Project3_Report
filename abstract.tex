

\begin{abstract}
    % Motivation
    Radiomics is the field of medicine where high amounts of quantitative features is
    extracted from medical images. The field is most studied in terms of
    oncology, where features are aggregated in tumors. The goal is to establish
    links between diagnosis, prognosis and treatment response, from
    correlations in quantitative radiomics features. However, image modalities
    such as Positron Emission Tomography (PET) suffers from high feature
    variability inter scanners and due to different image reconstruction parameters, etc.  
    In order for radiomics to have any clinical relevance, features values need
    to be harmonized and standardized. A feature values should reflect the
    underlying pathology and not vary due to scanner specific settings. It's
    therefore important to establish how different scanner parameter affect the
    feature value. If a link is established it may be possible to standardize
    certain features.  

    Feature selection is an important step in the machine learning process because it helps to reduce the complexity of the 
    model, improve the accuracy of the model, reduce overfitting, reduce
    the training time of the model, increase explain ability of the model.  Feature selection helps 
    to identify the most important features that contribute to the prediction
    and discard the irrelevant or redundant features. 

    % Especially in medicine it's important to be able to explain the outcomes of
    % a prediction.
    % What we have done

    Our dataset contains 15 radiomics features extracted on 16 different lymphom
    cancers acquired with a PET scanner. Each image is reconstructed with tree
    different reconstruction methods (our class label targets). We will try to
    identify the features that are best able the separate each reconstruction
    method in a systematic way. In the first part we will use random forest to
    identify the features with best class separability between reconstruction.
    Then we will apply feed forward selection with Suport Vector Machine (SVM)
    to see if we are able to correctly predict the class targets. And also
    validate if or prior features selection agrees with the
    results obtained with SVM.         


    % Main results





    % Implications
    % Try to establish models that maps recons?

\end{abstract}



