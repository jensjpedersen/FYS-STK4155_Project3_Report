%%%%%%%%%%%%%%%%%%%%%%%%%%%%%%%%%%%%%%%%%%%%%%%%%%%%%%%%%%%%%%%%%%%%%%%%%%%%%%%%%%%%%%%%%%
\section{Conclusion}

% Discussion of one predictor. 
From our feature selection analysis we identified the 5 best features in terms
of impurity decrease. The top features was histogram min, glcm entropy, glcm
contrast, histogram std and glcm homogeneity. 

We then predicted the test accuracy with SVM. Classification of the target
labels PT-PET-EARLAC vs PT-PET-WB-Q-CLEAR and PT-PET-EARL2 vs PT-PET-EARLAC
with only one predictor, produced the best accuracy for the features identified
in our prior feature selection analysis. 
Classification of PT-PET-EARLAC vs PT-PET-BB-Q-CLEAR produced an accuracy score
in the range 0.711-0.745. 
Classification of PT-PET-EARL2 vs PT-PET-EARLAC produced an accuracy score in
the range 0.644-0.680 for the above five features. 

Our feature selection method was
effective in selecting the best features for our SVM when only one predictor
was considered. 
This was not the case for classification of the targets PT-PET-EAR2 vs
PT-PET-WB-Q-CLEAR, which generally performed poorly on all the features. 

% Future improvements 
% Study if standardization is possible, yes/no? Based on accuracy 

% Was choice of features selection method effective? Was the results with the
% SVM as expected?


% Main results

% Discussion
The best accuracy obtained with SVM was classification of \verb|PT_PET_EARLAC| versus \verb|PT_PET_WB_Q_CLEAR|
using only two features \verb|histogram min| and \verb|glcm homogenity| which provided 
an accuracy of $0.817$. 

% The results classifying \verb|PT_PET_EARL2| versus \verb|PT_PET_EARLAC| showed 
% very similar results but with lower accuracy. 

The optimal accuracy gotten for classification of  \verb|PT_PET_EARL2| versus
\verb|PT_PET_EARLAC| was $0.713$, obtained using the same feature pair. 
This was one of the feature pair combinations we expected to produce the best
results with SVM, based on our prior feature selection analysis (see table \ref{tab:expectation})

Both classification combinations showed worse accuracy for every 
third feature added to the optimal pair of features. 


The results from the last classification combination
 \verb|PT_PET_EARL2| versus \verb|PT_PET_WB_Q_CLEAR| was quite different. 
 We observed a more fine tuned 
 parameter selection was needed as we got more predictive power with the lowest accuracies by predicting the 
 opposite of the model. By this method we found \verb|shape convex_hull_area| and \verb|histogram max| 
 to be the feature pair giving the best accuracy $0.752$, although the analysis of this classification 
 should be repeated with better tuned parameters.  


 %The results from the last classification combination
 %\verb|PT_PET_EARL2| versus \verb|PT_PET_WB_Q_CLEAR| performed poorly on all
 %features. 
 %We suspect that is due to    

 Comparing the results from different classification combinations, we hypothesized the \verb|PT_PET_EARLAC| 
 series to be most different from the other series. 

 % We also saw SVM was able to distinguish every series 
 % suggesting there are transformations able to improve the similarity in every series. 

 We successfully was able to user feature selection techniques to identify the
 features that produced the best class separation with SVM.     

The \verb|histogram min| and \verb|glcm homogenity| features produced the best
class separation between reconstruction methods.     
When considering such features in a raiomics setting it's important to be
aware that there will be significant variation in those feature values with
respect to different reconstruction methods
From the comparison of our accuracy score we conclude that    
\verb|PT_PET_EARLAC| and \verb|PT_PET_WB_Q_CLEAR|
is the least similar reconstruction methods  in terms of feature values.   
And \verb|PT_PET_EARL2| versus \verb|PT_PET_WB_Q_CLEAR| most similar. 

% XXX: write about results features 

% class separability - poor methodological approach
To use supervised learning methods to describe class separability is not a
good methodological approach. The purpose of supervised classification
method is to correctly identify class labels from an unseen dataset.     
The problem of exploring class separability and choice of dataset is
related to Jens J. Pedersen's master thesis. Thus, the motivation for the 
unconventional use of M.L. Methods. In a medical setting the labels of the
VOI'S and Reconstruction methods is always known. Therefore, simpler methods
should be utilized when analysing differences between PET reconstruction
methods. We did not manage to find any relevant literature on how the different features compares
with respect to the different series. Thus, the validity of the results is
uncertain. Our feature selection methods is relevant in conjunction with SVM and
will reduce the computational cost associated with training a model on a large
amount of features. Our methodological approach for exploring class separation
should not be explored any further. 
 

% Features selection


%  best serie 1 og 2  min og glcm homogenity -  acc  test 0.817 og 0.812
% do not f 

% 
