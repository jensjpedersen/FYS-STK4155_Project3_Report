%%%%%%%%%%%%%%%%%%%%%%%%%%%%%%%%%%%%%%%%%%%%%%%%%%%%%%%%%%%%%%%%%%%%%%%%%%%%%%%%%%%%%%%%%%
\section{Conclusion}

% Discussion of one predictor. 
From our feature selection analysis we identified the 5 best features in terms
of impurity decrease. The top features was histogram min, glcm entropy, glcm
contrast, histogram std and glcm homogeneity. 

We then predicted the test accarcy with SVM. Classification of the target
labels PT-PET-EARLAC vs PT-PET-WB-Q-CLEAR and PT-PET-EARL2 vs PT-PET-EARLAC
with only one predictor, produced the best accarcy for the features identfied
in our prior feature selection analysis. 
Classification of PT-PET-EARLAC vs PT-PET-WB-Q-CLEAR proudced an accarcy score
in the range 0.711-0.745. 
Classification of PT-PET-EARL2 vs PT-PET-EARLAC produced an accarcy score in
the range 0.644-0.680 for the above five features. 

Our feature selection method was
effective in selecting the best features for our SVM when only one predictor
was considered. 
This was not the case for calssification of the targets PT-PET-EAR2 vs
PT-PET-WB-Q-CLEAR, which generaly performed porly on all the features. 

% Future improvements 
% Study if standardization is possible, yes/no? Based on accuracy 

% Was choice of features selection method effective? Was the results with the
% SVM as expected?


% Main results

% Discussion
Our best accuracy obtained with SVM on the test data and k-fold re sampling
was XXX: score.
% XXX: write about results features 

% class separability - poor methodological approach
To use supervised learning methods to describe class separability is not a
good methodological approach. The purpose of supervised classification
method is to correctly identify class labels from an unseen dataset.     
The problem of exploring class separability and choice of dataset is
related to Jens J. Pedersen's master thesis. Thus, the motivation for the 
unconventional use of M.L. Methods. In a medical setting the labels of the
VOI'S and Reconstruction methods is always known. Therefore, simpler methods
should be utilized when analysing differences between PET reconstruction
methods. We did not mange to find any relevant literature on how the different features compares
with respect to the different series. Thus, the validity of the results is
uncertain. 
 

% Features selection


%  best serie 1 og 2  min og glcm homogenity -  acc  test 0.817 og 0.812
% do not f 

% 
