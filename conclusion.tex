%%%%%%%%%%%%%%%%%%%%%%%%%%%%%%%%%%%%%%%%%%%%%%%%%%%%%%%%%%%%%%%%%%%%%%%%%%%%%%%%%%%%%%%%%%
\section{Conclusion}

% Future improvements 
% Study if standardization is possible, yes/no? Based on accuracy 

% Was choice of features selection method effective? Was the results with the
% SVM as expected?


% Main results

% Discussion
Our best accuracy obtained with SVM on the test data and k-fold re sampling
was XXX: score.
% XXX: write about results features 

% class separability - poor methodological approach
To use supervised learning methods to describe class separability is not a
good methodological approach. The purpose of supervised classification
method is to correctly identify class labels from an unseen dataset.     
The problem of exploring class separability and choice of dataset is
related to Jens J. Pedersen's master thesis. Thus, the motivation for the 
unconventional use of M.L. Methods. In a medical setting the labels of the
VOI'S and Reconstruction methods is always known.  

% Features selection
