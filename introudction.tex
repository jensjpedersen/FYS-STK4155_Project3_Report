%%%%%%%%%%%%%%%%%%%%%%%%%%%%%%%%%%%%%%%%%%%%%%%%%%%%%%%%%%%%%%%%%%%%%%%%%%%%%%%%%%%%%%%%%%
\section{Introduction}

% Motivation for feature selection
Feature selection is an important step in the machine learning process because
it helps reduce the complexity of the model and helps improve its accuracy. By
selecting the most relevant features, the model can better focus on the
important data points and ignore the noise. This can increase the accuracy of
the model and help reduce overfitting. Additionally, feature selection helps
reduce the time and resources required to train the model, which can be
especially helpful when dealing with large datasets.
Selecting good features will make the model more interpretable by removing redundant
or irrelevant features. 

% Radiomics and identification of features  
Radiomics is a field of medical imaging that uses advanced
image processing algorithms to extract quantitative features from medical
images such as CT, MRI, and PET scans. These features are then used to develop
predictive models to better understand a patient's disease and to predict
outcomes. Radiomics can be used to better understand the biology of cancer, to
develop personalized treatments, and to monitor the effectiveness of
treatments.

% Radiomics problem 
There are some serious challenges that needs to be solved in order for
Radiomics to be clinically relevant.     
A Radiomics feature should reflect the underlying pathology and not vary due to
changes in scanner type or scanner specific parameters. Different scanner
types, reconstruction and image processing parameters affect feature values
differently. These parameters need to be controlled in order for the field to have any
clinical relevance. 


% There is currently a high priority in the field to identify methods for
% harmonization features values. 

% TODO: write harmonization
% Feature harmonization in radiomics refers to the process of standardizing the
% calculation and representation of radiomic features across different imaging
% studies or datasets. This is important because radiomic features can be
% affected by various factors such as imaging modality, acquisition parameters,
% and image processing techniques, which can lead to significant variability in
% the calculated features even within a single study.

% To address this variability, feature harmonization involves establishing
% standardized protocols for calculating and representing radiomic features that
% can be applied consistently across different studies or datasets. 

By harmonizing the calculation and representation of radiomic features, it is
possible to more accurately compare and analyze the features across different
studies or datasets, enabling more reliable and reproducible results in
radiomics research.

% TODO: feature selection
% TODO: report structure
% robust with respect to scanner type and scanner specific parameters. Also
% different harmonization techniques    

% Problem non robust feature

% Dataest
In this report we will explore a lymphoma image dataset acquired with a PET % XXX add model
scanner. A total of n = 16 malignant tumors was identified in the image data 
Each image is reconstructed with different reconstruction protocols (our class
labels). A total of 15 radiomics features is extracted from the tumours (Volume
of Interest) 
% Objective: discriminate between reconstruction
We will explore simple machine learning algorithms to explore if it's possible
to discriminate between the different reconstruction methods. 

If a clear pattern between reconstruction algorithms exists, it may be possible to
transform features values between different reconstruction methods. And
therefore control for the specific reconstruction methods. In such a way that
it may be possible to define absolute standards for the interpretation of a specific
feature value and it's significance in terms of pathology. We will not try to
harmonize the data, but rather explore if harmonization models it's worth
exploring.  

% In this project we will use simple supervised machine learning algorithms to
% identify  

% What we have done
In the first part of the report we will implement simple feature selection
techniques to identify features that are able to discriminate between the
different PET reconstruction methods. We will use random forest to identify the
features with best discriminative power. Then we will use person correlation to
identify highly correlated redundant features. In the last part we will
use feed forward feature selection in conjugation with Support Vector Machine
(SVM). We will then evaluate if our prior analysis is in agreement with the
results obtained with SVM. If our models and selected features is able to
accurately discriminate between the different classes. Then, it may be possible
find models that are able to map feature values between series, and define
absolute standards for the quantitative meaning of a feature value with respect
to pathology. 

% Report structure
% Method 
The first part of the method section (\ref{sec:method_dataset}) explains how the data is organized and
how to reproduce the data with python code. Subsection
\ref{sec:method_descion_tree} and \ref{sec:method_random_forest} contains the
necessary theory on how to implement a random forest. The last part of
subsection \ref{sec:method_random_forest} explains how to reproduce our results
obtained with random forest feature selection. And also comparison of our own
implementation vs the sci-kit learn python library. The code used to produce
our results with the Random Forest algorithm can be found in the Github repository,
\url{https://github.com/jensjpedersen/Projects_FYS-STK4155/tree/main/Project3} 

% TODO Method SVM 
In \autoref{sec:SVM}, first the theory of Support Vector Machines is layed out and explained
in \autoref{sec:SVM_theory}. Then a code example is added to show how the SVM optimization problem might be solved
and an explanation of how we will utilize the SVM is given in \autoref{sec:SVM_method}. 
The code used to produce our results with SVM can be found in the Github repository,
\url{https://github.com/fredrikjp/FYS-STK4155/tree/master/Project3}. 



% Results random forest
The first part of our Discussion we present our results and discussion from
feature selection with Random Forest and Correlation analysis.  
Then the results and discussion from our Support Vector Machine is presented for 
different series classifications. 

Finally we summarize and conclude the project in the Conclusion section.  
% Results SVM 



