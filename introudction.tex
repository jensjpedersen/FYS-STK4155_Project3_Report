%%%%%%%%%%%%%%%%%%%%%%%%%%%%%%%%%%%%%%%%%%%%%%%%%%%%%%%%%%%%%%%%%%%%%%%%%%%%%%%%%%%%%%%%%%
\section{Introduction}




\paragraph{Purpose} 
\begin{itemize}
    \item Implement ML algorithms for automatic feature reduction in classification
        problem
    \item Use non-automatic methods; correlation matrix, simple similarity
        measures to select features with best separability   

    \item Compare Results from non-automatic and automatic methods
\end{itemize}



\begin{itemize}
    \item Explore ML algorithms suited for classification on spares data  
    \item Feature reduction and find optimal set of feature for classification
    \item Classification Data: 
        \begin{itemize}
            \item Cancer data with 17 tumors
            \item Each tumor is reconstructed with 3 different scanner
                reconstruction methods, our classes 
            \item Thus, tot. data = 17*3
            \item Different features is extracted from images   
        \end{itemize}
        
\end{itemize}


\paragraph{Approach} 
\paragraph{Manual feature selection} 
\begin{itemize}
    \item First we will use visualization techniques to identify good features.
        That is features with good class separability, non-correlated features,
        features with high correlation with target. 
    \item teqniques: anova, pearson-correlation, 
    \item Expertise: Volume is useless
\end{itemize}

\paragraph{Automatic selection - Random forest} 
\begin{itemize}
    \item bagging
\end{itemize}

\paragraph{Automatic selection - SVM} 
\begin{itemize}
    \item Use wrapper methods such as forward selection to find optimal feature
        combinations
    \item Compare Features from SVM with previous methods
    \item Significance testing of results
\end{itemize}


% Radiomics and identification of features  
Radiomics is a field of medical imaging that uses advanced
image processing algorithms to extract quantitative features from medical
images such as CT, MRI, and PET scans. These features are then used to develop
predictive models to better understand a patient's disease and to predict
outcomes. Radiomics can be used to better understand the biology of cancer, to
develop personalized treatments, and to monitor the effectiveness of
treatments.


There are some serious challenges that needs to be solved in order for
Radiomics to be clinically relevant.     
A Radiomics feature should reflect the underlying pathology and not vary due to
changes in scanner type or scanner specific parameters. Different scanner
types, reconstruction- and (XXX:bildepalass) -parameters affect feature values
differently. These parameters needs to be controlled for the field to have any
clinical relevance. 
There is currently a high priority in the field to identify features that are

% TODO: write harmonization
% TODO: feature selection
% TODO: report structure
% robust with respect to scanner type and scanner specific parameters. Also
% different harmonization techniques    

% Problem non robust feature


% Dataest
In this report we will explore a lymphoma image dataset acquired with a PET % XXX add model
scanner. A total of n = 16 malignant tumors was identified in the image data 
Each image is reconstructed with different protocols (our class
labels). A total of 15 radiomics features is extracted from the tumours (Volume
of Interest) 
% Objective: discriminate between reconstruction
We will explore simple machine learning algorithms to explore if it's possible
to discriminate between the different reconstruction methods. Hopefully some of
the difference seen in the different reconstruction methods will resamble
differences with respect to different scanner types.    

If a clear pattern between reconstruction algorithms exists, it may be possible to
transform features between reconstruction method. In such a way that we can find an 
absolute standard for what a feature value means in terms of pathology. 


% Purpose identify   
In this project we will use simple supervised machine learning algorithms to
identify  


% Motivation for feature selection
Feature selection is an important step in the machine learning process because
it helps reduce the complexity of the model and helps improve its accuracy. By
selecting the most relevant features, the model can better focus on the
important data points and ignore the noise. This can increase the accuracy of
the model and help reduce overfitting. Additionally, feature selection helps
reduce the time and resources required to train the model, which can be
especially helpful when dealing with large datasets.

Feature selection reduces overfitting by helping to identify the most relevant
features that are important for a model to learn from. It can help to reduce
the complexity of a model and make it more interpretable by removing redundant
or irrelevant features. This can also help to reduce the amount of noise in the
data, making it easier for the model to generalize to unseen data.




% What we have done
In the first part of the report we will implement simple feature selection
techniques to identify features that are able to discriminate between the
different PET reconstruction methods. We will use random forest to identify the
features with best discriminative power. Then we will use person correlation to
identify highly correlated redundant features, and ANOVA to identify
features that is highly correlated with the targets classes. In the last part we will
use feed forward feature selection in conjugation with Support Vector Machine
(SVM). We will then evaluate if our prior analysis is in agreement with the
results obtained with SVM. And also  Our own implementations of the algorithms will be
tested with the scikit learn python package. 

Is our prior analysis as expected. 




